\documentclass[a4paper,pdftex]{article}

\usepackage{array}
\usepackage{mathtools}
\usepackage{hopsantut}

\hypersetup{pdfauthor={Robert Braun and Peter Nordin}, pdftitle={Hopsan Tutorial - Getting Started}, pdfsubject={Hopsan Tutorial}}

\begin{document}
\maketitle{Writing Component Libraries}

\section{Requirements}
%Man behöver Hopsan och HoLC

\section*{Hopsan Component Libraries}
Hopsan component libraries are written in C++ and compiled to shared library files (.dll in Windows or .so in Linux). A Hopsan component library constists of the files listen below. First, there is a \textit{library description file} in .xml format, which contains basic information about the library. Second, there is a source code file of the library with the .cc extension.\vspace{10pt}\\
$\mathrm{Component\ Library} \begin{cases}
\mathtt{\textbf{MyLibrary.xml}} \mathrm{\ -\ Library\ description\ file}\\
\mathtt{\textbf{MyLibrary.cc \mathrm{\ -\ Library\ source\ code}}\vspace{7pt} \\ 
\begin{rcases}
  \mathtt{MyFirstComponent.hpp} \\
  \mathtt{MyFirstComponent.xml} \\
  \mathtt{MyFirstComponent.svg}
\end{rcases} \mathrm{Component}\vspace{7pt} \\
\begin{rcases}
  \mathtt{MySecondComponent.hpp} \\
  \mathtt{MySecondComponent.xml} \\
  \mathtt{MySecondComponent.svg}
\end{rcases} \mathrm{Component} \\
\end{cases}$

\subsection*{Library Description File}
%Beksriv lib.xml

\subsection*{Library Source File}
%Beskriv lib.cc

\subsection*{Component Source Files}
%Beskriv comp.hpp

\subsection*{Component Appearance Files}
%Beskriv comp.xml

\subsection*{Component Icon Files}
%Beskriv att man kan göra ikoner med Inkscape

\section{Example}
%Beskriv hur man bygger ett bibliotek med en enkel aritmetisk komponent i HoLC

\begin{enumerate}
\item \textbf{Title 1}\\
Description 1

\item \textbf{Title 2}\\
Description 2

\icon{0}{gfx/Hopsan-New.png}{Create a new empty model}
\end{enumerate}
 	
\end{document}